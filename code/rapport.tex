\documentclass[a4paper]{article}

\usepackage[T1]{fontenc}
\usepackage[utf8]{inputenc}
\usepackage[french]{babel}
\usepackage{mathpazo}
\usepackage[scaled]{helvet}
\usepackage{courier}
\usepackage[sf,bf]{titlesec}
\usepackage[margin=0.75in]{geometry}
\usepackage{tabularx}

\makeatletter
\newenvironment{expl}{%
  \begin{list}{}{%
      \small\itshape%
      \topsep\z@%
      \listparindent0pt%\parindent%
      \parsep0.75\baselineskip%\parskip%
      \setlength{\leftmargin}{20mm}%
      \setlength{\rightmargin}{20mm}%
    }
  \item[]}%
  {\end{list}}
\makeatother

% Indiquez votre prénom (en minuscules) et votre nom (en majuscules)
\title{Rapport de TP \\ Architecture des systèmes d'exploitation}
\author{\underline{Prénom} \underline{NOM}}
\date{Novembre 2021}

\begin{document}

\maketitle

% Mettez une table des matières si votre rapport contient plus
% de 3 pages ou si vous ne suivez pas le plan suggéré :
%\tableofcontents

% Dans votre rapport final, supprimez toutes les explications
% (c'est-à-dire tous les environnements \begin{expl} ... \end{expl}).
\begin{expl}
  Ce document vous suggère une trame de rapport de TP.

  Vous pouvez utiliser le source \LaTeX{} de ce document comme
  base (voir également la cible \texttt{rapport.pdf} dans le fichier
  \texttt{Makefile}), mais vous pouvez utiliser tout autre outil
  de votre choix, à condition de rendre un rapport en PDF.
\end{expl}

\section{Introduction}

\begin{expl}
  Rappelez brièvement le sujet et esquissez les principes de votre
  solution (une dizaine de lignes maximum).
\end{expl}

\section{Structure de données}

\subsection{Structures de données partagées}\label{sec-shm}

\begin{expl}
  Décrivez ici la structure de votre segment de mémoire partagée,
  ainsi que toutes les structures annexes que vous y utilisez. Pour
  chaque donnée, indiquez son nom, sa nature (entier, sémaphore,
  tableau de X...), sa valeur initiale, et décrivez en quelques
  mots son usage.

  Par exemple, si le vaccinodrome disposait d'un distributeur de gel
  hydroalcoolique à l'usage des patients, on trouverait la description
  suivante~:

  \begin{tabularx}{\linewidth}{|l|l|l|X|}
    \hline
    % \multicolumn{4}{|l|}{Distributeur de gel
    %   (\texttt{struct distr\_gelha})}
    % \\ \hline
    Champ & Type & Init & Description \\ \hline%
    dg & distrib\_t & -- & distributeur (cf. documentation de
    \texttt{libhygiene}) \\ \hline%
    sm\_dg & asem\_t & 1 & sémaphore utilisé pour synchroniser l'accès
    au distributeur \\ \hline%
  \end{tabularx}

  On considère que tout ce qui est décrit dans cette section existe
  depuis la création du vaccinodrome jusqu'à sa fermeture.
\end{expl}

\subsection{Structures de données non partagées}

\begin{expl}
  Indiquez toutes les données conservées par les processus mais non
  partagées. La forme est la même que dans la section précédente.
  Précisez aussi à quel moment ces données sont créées et
  initialisées.
\end{expl}

\section{Synchronisations}

\begin{expl}
  Chacune des sous-sections suivantes doit décrire un aspect de la
  synchronisation. Elles ont toutes la même organisation, et doivent
  contenir deux ou trois parties clairement délimitées.

  \begin{enumerate}

  \item La première partie décrit tous les objets impliqués dans la
    synchronisation de l'accueil des patients dans le vaccinodrome. Il
    s'agit des données «~normales~» (par exemple un entier) et des
    sémaphores. Pour chacun de ces éléments, précisez si il est
    rattaché au vaccinodrome globalement ou à un patient, ou encore à
    un médecin. Par exemple, si le vaccinodrome était muni d'un
    distributeur de gel hydroalcoolique grâce auxquels les patients
    peuvent se nettoyer les mains, il faudrait mentionner~:

    \begin{tabularx}{\linewidth}{|l|l|>{\strut}X|}
      \hline%
      sm\_dg & asem\_t & le sémaphore représentant le nombre de
      distributeurs \underline{du vaccinodrome} (voir
      section~\ref{sec-shm}) \\ \hline%
      monid & int & le nom %
      \underline{\raisebox{0pt}[\height][0pt]{du patient}} \\ \hline%
    \end{tabularx}

  \item La seconde partie doit contenir un pseudo-code de ce
    qu'exécute \emph{chaque} acteur impliqué (le patient, le médecin,
    autres). Seuls les aspects importants doivent être présentés~:
    opérations sur les sémaphores, affectation de valeurs aux
    variables partagées... Le reste (par exemple les messages liés à
    \texttt{DEBUG\_OUTPUT}) ne doivent pas apparaître ici. Dans
    l'exemple des distributeurs de gel, le pseudo-code serait~:

\begin{verbatim}
// Ce code est exécuté par le patient
P (sm_dg)
se_laver_les_mains (monid)
V (sm_dg)
\end{verbatim}

  \item La troisième partie contient toutes les remarques que vous
    jugez pertinentes. Vous pouvez par exemple y préciser les
    conditions de concurrence, ou les éventuelles limitations ou
    propriétés remarquables de votre solution.

    Dans l'exemple des distributeurs de gel, vous pourriez préciser
    que vous avez pensé à implémenter à la place une attente limitée
    dans le temps pour les patients qui portaient des gants en entrant
    (en expliquant comment vous auriez fait), mais que cela aurait
    compliqué le code pour un intérêt limité.
  \end{enumerate}

  Vous pouvez ajouter des sous-sections si vous avez d'autres
  synchronisations importantes, ou fragmenter celles qui vous sont
  proposées ci-dessous.
\end{expl}


\subsection{Arrivée d'un patient}\label{arrivee-patient}

\begin{expl}
  Expliquez dans cette sous-section, conformément aux indications
  ci-dessus, tout ce qui se passe lorsqu'un patient arrive,
  \emph{avant} qu'il interagisse avec un médecin (si le vaccinodrome
  est ouvert) ou reparte. Mentionnez toutes les informations
  partagées qui sont utilisées et toutes les synchronisations
  nécessaires.
\end{expl}

\subsection{Arrivée d'un médecin}

\begin{expl}
  Expliquez dans cette sous-section tout ce qui se passe lorsqu'un
  médecin arrive, \emph{avant} qu'il interagisse avec un patient.
\end{expl}

\subsection{Interactions entre patients et médecins}

\begin{expl}
  Expliquez ici comment patients et médecins sont «~mis en contact~».
  Expliquez la stratégie que vous avez choisie, c'est-à-dire comment
  1) soit un patient trouve un médecin dès qu'un médecin est
  disponible, 2) soit un médecin trouve un patient dans la salle
  d'attente. Expliquez également ce qui se passe entre le moment où un
  médecin et un patient sont mis en contact, et le moment où le
  patient vacciné peut quitter le vaccinodrome et le médecin s'occuper
  d'un (éventuel) autre patient.
\end{expl}

\subsection{Fermeture du vaccinodrome}

\begin{expl}
  Expliquez les détails de ce que fait le programme \texttt{fermer}
  (et seulement ce programme), y compris lorsqu'il reste des patients
  et/ou des médecins.
\end{expl}

\subsection{Patients après fermeture}

\begin{expl}
  Expliquez ce qui se passe pour un patient assis dans la salle
  d'attente après la fermeture. (Notez que le cas du patient qui
  arrive \emph{après} la fermeture doit avoir été décrit dans la
  section~\ref{arrivee-patient}.)
\end{expl}

\subsection{Médecins après fermeture}

\begin{expl}
  Expliquez ce qui se passe pour un médecin après la fermeture. (Le
  cas du médecin qui arrive \emph{après} la fermeture doit avoir été
  décrit dans la section~\ref{arrivee-patient}.)
\end{expl}

\section{Remarques sur l'implémentation}

\begin{expl}
  Placez ici toute remarque relative à vos choix d'implémentation (si
  vous en avez). Par exemple~: si vous avez placé des temporisations
  particulières pour vérifier que certains cas de concurrence étaient
  gérés correctement, expliquez cela dans cette section. Toute autre
  remarque qui aide à comprendre votre code est à placer ici.
\end{expl}

\section{Conclusion}

\begin{expl}
  Tirez les conclusions de votre projet~: limites de votre
  implémentation, difficultés particulières, subtilités dont vous êtes
  fiers, etc.
\end{expl}

\end{document}
